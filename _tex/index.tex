% Options for packages loaded elsewhere
\PassOptionsToPackage{unicode}{hyperref}
\PassOptionsToPackage{hyphens}{url}
\PassOptionsToPackage{dvipsnames,svgnames,x11names}{xcolor}
%
\documentclass[
  letterpaper,
  DIV=11]{scrartcl}

\usepackage{amsmath,amssymb}
\usepackage{iftex}
\ifPDFTeX
  \usepackage[T1]{fontenc}
  \usepackage[utf8]{inputenc}
  \usepackage{textcomp} % provide euro and other symbols
\else % if luatex or xetex
  \usepackage{unicode-math}
  \defaultfontfeatures{Scale=MatchLowercase}
  \defaultfontfeatures[\rmfamily]{Ligatures=TeX,Scale=1}
\fi
\usepackage{lmodern}
\ifPDFTeX\else  
    % xetex/luatex font selection
\fi
% Use upquote if available, for straight quotes in verbatim environments
\IfFileExists{upquote.sty}{\usepackage{upquote}}{}
\IfFileExists{microtype.sty}{% use microtype if available
  \usepackage[]{microtype}
  \UseMicrotypeSet[protrusion]{basicmath} % disable protrusion for tt fonts
}{}
\makeatletter
\@ifundefined{KOMAClassName}{% if non-KOMA class
  \IfFileExists{parskip.sty}{%
    \usepackage{parskip}
  }{% else
    \setlength{\parindent}{0pt}
    \setlength{\parskip}{6pt plus 2pt minus 1pt}}
}{% if KOMA class
  \KOMAoptions{parskip=half}}
\makeatother
\usepackage{xcolor}
\setlength{\emergencystretch}{3em} % prevent overfull lines
\setcounter{secnumdepth}{5}
% Make \paragraph and \subparagraph free-standing
\makeatletter
\ifx\paragraph\undefined\else
  \let\oldparagraph\paragraph
  \renewcommand{\paragraph}{
    \@ifstar
      \xxxParagraphStar
      \xxxParagraphNoStar
  }
  \newcommand{\xxxParagraphStar}[1]{\oldparagraph*{#1}\mbox{}}
  \newcommand{\xxxParagraphNoStar}[1]{\oldparagraph{#1}\mbox{}}
\fi
\ifx\subparagraph\undefined\else
  \let\oldsubparagraph\subparagraph
  \renewcommand{\subparagraph}{
    \@ifstar
      \xxxSubParagraphStar
      \xxxSubParagraphNoStar
  }
  \newcommand{\xxxSubParagraphStar}[1]{\oldsubparagraph*{#1}\mbox{}}
  \newcommand{\xxxSubParagraphNoStar}[1]{\oldsubparagraph{#1}\mbox{}}
\fi
\makeatother


\providecommand{\tightlist}{%
  \setlength{\itemsep}{0pt}\setlength{\parskip}{0pt}}\usepackage{longtable,booktabs,array}
\usepackage{calc} % for calculating minipage widths
% Correct order of tables after \paragraph or \subparagraph
\usepackage{etoolbox}
\makeatletter
\patchcmd\longtable{\par}{\if@noskipsec\mbox{}\fi\par}{}{}
\makeatother
% Allow footnotes in longtable head/foot
\IfFileExists{footnotehyper.sty}{\usepackage{footnotehyper}}{\usepackage{footnote}}
\makesavenoteenv{longtable}
\usepackage{graphicx}
\makeatletter
\newsavebox\pandoc@box
\newcommand*\pandocbounded[1]{% scales image to fit in text height/width
  \sbox\pandoc@box{#1}%
  \Gscale@div\@tempa{\textheight}{\dimexpr\ht\pandoc@box+\dp\pandoc@box\relax}%
  \Gscale@div\@tempb{\linewidth}{\wd\pandoc@box}%
  \ifdim\@tempb\p@<\@tempa\p@\let\@tempa\@tempb\fi% select the smaller of both
  \ifdim\@tempa\p@<\p@\scalebox{\@tempa}{\usebox\pandoc@box}%
  \else\usebox{\pandoc@box}%
  \fi%
}
% Set default figure placement to htbp
\def\fps@figure{htbp}
\makeatother
% definitions for citeproc citations
\NewDocumentCommand\citeproctext{}{}
\NewDocumentCommand\citeproc{mm}{%
  \begingroup\def\citeproctext{#2}\cite{#1}\endgroup}
\makeatletter
 % allow citations to break across lines
 \let\@cite@ofmt\@firstofone
 % avoid brackets around text for \cite:
 \def\@biblabel#1{}
 \def\@cite#1#2{{#1\if@tempswa , #2\fi}}
\makeatother
\newlength{\cslhangindent}
\setlength{\cslhangindent}{1.5em}
\newlength{\csllabelwidth}
\setlength{\csllabelwidth}{3em}
\newenvironment{CSLReferences}[2] % #1 hanging-indent, #2 entry-spacing
 {\begin{list}{}{%
  \setlength{\itemindent}{0pt}
  \setlength{\leftmargin}{0pt}
  \setlength{\parsep}{0pt}
  % turn on hanging indent if param 1 is 1
  \ifodd #1
   \setlength{\leftmargin}{\cslhangindent}
   \setlength{\itemindent}{-1\cslhangindent}
  \fi
  % set entry spacing
  \setlength{\itemsep}{#2\baselineskip}}}
 {\end{list}}
\usepackage{calc}
\newcommand{\CSLBlock}[1]{\hfill\break\parbox[t]{\linewidth}{\strut\ignorespaces#1\strut}}
\newcommand{\CSLLeftMargin}[1]{\parbox[t]{\csllabelwidth}{\strut#1\strut}}
\newcommand{\CSLRightInline}[1]{\parbox[t]{\linewidth - \csllabelwidth}{\strut#1\strut}}
\newcommand{\CSLIndent}[1]{\hspace{\cslhangindent}#1}

\KOMAoption{captions}{tableheading}
\makeatletter
\@ifpackageloaded{caption}{}{\usepackage{caption}}
\AtBeginDocument{%
\ifdefined\contentsname
  \renewcommand*\contentsname{Inhaltsverzeichnis}
\else
  \newcommand\contentsname{Inhaltsverzeichnis}
\fi
\ifdefined\listfigurename
  \renewcommand*\listfigurename{Abbildungsverzeichnis}
\else
  \newcommand\listfigurename{Abbildungsverzeichnis}
\fi
\ifdefined\listtablename
  \renewcommand*\listtablename{Tabellenverzeichnis}
\else
  \newcommand\listtablename{Tabellenverzeichnis}
\fi
\ifdefined\figurename
  \renewcommand*\figurename{Abbildung}
\else
  \newcommand\figurename{Abbildung}
\fi
\ifdefined\tablename
  \renewcommand*\tablename{Tabelle}
\else
  \newcommand\tablename{Tabelle}
\fi
}
\@ifpackageloaded{float}{}{\usepackage{float}}
\floatstyle{ruled}
\@ifundefined{c@chapter}{\newfloat{codelisting}{h}{lop}}{\newfloat{codelisting}{h}{lop}[chapter]}
\floatname{codelisting}{Listing}
\newcommand*\listoflistings{\listof{codelisting}{Listingverzeichnis}}
\makeatother
\makeatletter
\makeatother
\makeatletter
\@ifpackageloaded{caption}{}{\usepackage{caption}}
\@ifpackageloaded{subcaption}{}{\usepackage{subcaption}}
\makeatother

\ifLuaTeX
\usepackage[bidi=basic]{babel}
\else
\usepackage[bidi=default]{babel}
\fi
\babelprovide[main,import]{ngerman}
% get rid of language-specific shorthands (see #6817):
\let\LanguageShortHands\languageshorthands
\def\languageshorthands#1{}
\ifLuaTeX
  \usepackage[german]{selnolig} % disable illegal ligatures
\fi
\usepackage{bookmark}

\IfFileExists{xurl.sty}{\usepackage{xurl}}{} % add URL line breaks if available
\urlstyle{same} % disable monospaced font for URLs
\hypersetup{
  pdftitle={Vorgaben für PRX-Berichte},
  pdfauthor={Mirco Meiners},
  pdflang={de},
  pdfkeywords={Berichtswesen, Technischer Bericht},
  colorlinks=true,
  linkcolor={blue},
  filecolor={Maroon},
  citecolor={Blue},
  urlcolor={Blue},
  pdfcreator={LaTeX via pandoc}}


\title{Vorgaben für PRX-Berichte}
\author{Mirco Meiners}
\date{2024-11-27}

\begin{document}
\maketitle


\section{Technischer Bericht}\label{technischer-bericht}

Für das Erstellen eines Technischen Berichts finden Sie im nachfolgenden
Text Hinweise, die der Arbeit von Prof.~Dr. Boes entnommen wurden (Boes
2012).

Ziel eines Technischen Berichtes ist es, die Aufmerksamkeit des Lesers
auf das Wesentliche zu konzentrieren und dieses in einem logischen und
verständnisvollen Aufbau zu präsentieren. Die aufgeführten Unterlagen
und Inhalte sollen zu einem überzeugenden Ergebnis führen.
Entscheidungswege und Begründungen müssen ehrlich und nachvollziehbar
dargelegt werden.

Der Technische Bericht sollte in einem einfachen, konzentrierten Stil
geschrieben sein und Missverständnisse ausschliessen. Er soll für alle
Betroffenen sofort verständlich sein, auch wenn es sich nicht um
besonders versierte Fachleute handelt.

An dieser Stelle sei erwähnt, dass Berichte, wie sie im Rahmen von
Projektarbeiten, Bachelor- oder Masterarbeiten erstellt werden, nur
selten praxisgerechten Berichten entsprechen. Dies wird in der Regel
auch nicht verlangt.

\subsection{Bevor man mit dem Schreiben beginnen kann
\ldots{}}\label{bevor-man-mit-dem-schreiben-beginnen-kann}

\begin{enumerate}
\def\labelenumi{\arabic{enumi}.}
\item
  \textbf{Analyse der Aufgabenstellung}

  Folgende Fragen sollte man sich vor dem Lösen der Aufgabenstellung
  stellen:

  \begin{itemize}
  \item
    Was will der Auftraggeber?
  \item
    Ist der Auftrag/die Aufgabenstellung in der gegebenen Form sinnvoll?
  \item
    Wie weit darf ich von dem an mich gerichteten Auftrag abweichen?
  \item
    In der Praxis sollte man davon absehen, den Auftrag zu verändern.
    Bei wissenschaftlichen Arbeiten erscheint es im Lauf der Arbeit oft
    sinnvoll, die Aufgabenstellung anzupassen. Anpassungsvorschläge sind
    in jedem Falle mit der Betreuung zu besprechen.
  \end{itemize}
\item
  \textbf{Arbeitsprogramm und Zeitplan}

  Jede Arbeit muss geplant werden. Dazu muss vorher klar sein, aus
  welchen verschiedenen Aktivitäten sich die Arbeit in etwa
  zusammensetzt. Sinnvoll ist es, schon am Anfang eine Grobgliederung
  des Aufbaus (erstes Inhaltsverzeichnis) zu erstellen. Nachdem die
  einzelnen Aktivitäten definiert worden sind, können sie auf einer
  Zeitskala aufgetragen werden. Dabei sollte parallel zur Erarbeitung
  von Lösungen der Aufgabenstellung genügend Zeit für die Verfassung des
  Technischen Berichtes eingeplant werden. Zeitpläne werden oft zu
  optimistisch bemessen. Insbesondere die Schlussredaktion braucht oft
  viel Zeit. Darunter leidet oft eine seriöse Durchsicht. Deshalb
  sollten unbedingt Pufferzeiten eingeplant werden. Ein Zeitplan ist
  sinnvoll, wenn auch eine Fortschrittskontrolle durchgeführt wird.
\item
  \textbf{Literaturrecherche}

  Der Literaturbeschaffung kommt in einer wissenschaftlichen Arbeit mehr
  Bedeutung zu, als bei einer Arbeit mit Projektierungsschwerpunkt.
  Wichtig ist eine ziel- bzw. problemorientierte Literaturrecherche.
  Informationen zum Thema findet man z.B. in Zeitschriftenartikeln,
  Buchveröffentlichungen, Tagungsbänden, Dissertationen, Studien- und
  Diplomarbeiten und dem Internet. Die Assistenz für Wasserbau verfügt
  eine umfassende Fachartikelsammlung zu diversen Themen, in welche von
  den Studierenden auf Anfrage eingesehen werden kann.
\end{enumerate}

\section{Vorgaben}\label{vorgaben}

Liebe Studierende,

entsprechend der Bachelor-Prüfungsordnung müssen Sie einen
Arbeitsbericht zum Praxissemester (Modul PRX) abgeben. Um Ihnen
Nachfragen zu ersparen und die Abfassung zu erleichtern, gibt es für
diesen Bericht Vorgaben. Der Praxissemesterbericht ist in zwei Teilen
vorzulegen:

\begin{enumerate}
\def\labelenumi{\arabic{enumi}.}
\item
  \textbf{Zwischenbericht:} Abgabe spätestens vier Wochen nach Aufnahme
  der Praxissemestertätigkeit.
\item
  \textbf{Abschlussbericht:} Abgabe spätestens zwei Wochen nach
  Beendigung der Praxissemestertätigkeit.
\end{enumerate}

Für beide Berichte gilt:

\begin{itemize}
\item
  Es gibt ein Titelblatt mit Überschrift der Art des Berichtes,
  Zwischen- bzw. Abschlussbericht, über das Praxissemester von - bis und
  in welcher Firma. Das Titelblatt soll mindestens Ihren Namen, Ihre
  Matrikelnummer und E-Mail-Adresse sowie Name, E-Mail und Telefonnummer
  der betreuenden Person in der Firma aufführen.
\item
  Der Bericht soll ein Inhaltsverzeichnis enthalten. Die Seiten sind zu
  nummerieren. Bei Zitaten ist ein Verzeichnis der Quellen angebracht.
\item
  Es gibt keine Vorgabe für die Anzahl der Seiten. Der Bericht soll
  kurz, klar und verständlich die wesentlichen Inhalte darstellen. Eine
  Erhöhung der Seitenzahl durch langatmige Beschreibungen oder sehr
  große Schrift ist nicht erwünscht.
\item
  Der Bericht wird elektronisch in der entsprechenden AULIS-Gruppe
  abgelegt.
\item
  Sie finden Vorlagen im entsprechenden AULIS-Ordner.
\end{itemize}

\subsection{Der Zwischenbericht}\label{der-zwischenbericht}

\begin{itemize}
\item
  Eine kurze Beschreibung des Unternehmens.
\item
  Die Beschreibung Ihres Arbeitsbereichs bzw. Ihrer Abteilung.
\item
  Ihre Aufgabenstellung im Praxissemester.
\item
  In der Regel soll der Zwischenbericht nicht mehr als 10 Seiten
  umfassen.
\end{itemize}

\subsection{Der Abschlussbericht}\label{der-abschlussbericht}

\begin{itemize}
\item
  Tätigkeitsbericht, inhaltlich oder zeitlich gruppiert. Aus diesem
  Bericht soll genau hervorgehen, was Ihr eigener persönlicher Anteil an
  den zu erledigenden Aufgaben war.
\item
  Besondere Ergebnisse und Erfolge Ihrer Tätigkeit.
\item
  Bewertende Zusammenfassung des Praxissemesters
\item
  Der Abschlussbericht soll Inhalte des Zwischenberichts nur soweit
  wiederholen, wie sie zum Verständnis des Berichts notwendig sind. Die
  Aufnahme von Tages- bzw. Wochenberichten in Listenform in den
  Abschlussbericht ist nicht erforderlich. In der Regel soll der
  Abschlussbericht nicht mehr als 30 Seiten umfassen.
\end{itemize}

\textbf{Viel Erfolg!}

\section{Tipps zum Schreiben eines
Berichtes}\label{tipps-zum-schreiben-eines-berichtes}

\subsection{Verfassen eines ersten
Entwurfs}\label{verfassen-eines-ersten-entwurfs}

Allgemein sind beim Schreiben des ersten Entwurfs folgende Punkte zu
beachten:

\begin{itemize}
\item
  Die wichtigsten Punkte in einem Kapitel sollten in einem Zug
  geschrieben werden. Will dies nicht gelingen, wird empfohlen, zu einem
  nächsten Kapitel überzugehen, damit man trotzdem zügig vorankommt.
\item
  Schnell schreiben, ohne sich um fehlerhafte Formulierungen oder
  schlecht durchdachte Ausdrücke zu kümmern.
\item
  Kurze Sätze verwenden!
\item
  Bilder, Grafiken und Tabellen evtl. noch gar nicht einfügen (nur
  vermerken)
\end{itemize}

\subsection{Häufige Fehler}\label{huxe4ufige-fehler}

\begin{itemize}
\tightlist
\item
  Verwenden von \emph{ich} oder \emph{wir}
\item
  Substantivismus \emph{Beamtendeutsch}
\item
  Schachtelsätze
\item
  Füllwörter (Verlegenheitskonjunktionen)
\item
  Passiv-Konstruktionen
\item
  Helvetismen
\item
  Mode- und Fremdwörter
\item
  Uneinheitliche Quellenverweise im Text
\item
  Mangelnde Quellenverweise und mangelnde Verweise zum Anhang, zu
  Plänen, etc.
\end{itemize}

\subsection{Abkürzunge}\label{abkuxfcrzunge}

Abkürzungen sollen nur dort angewandt werden, wo Klarheit und Lesbarkeit
des Textes nicht beeinträchtigt werden. Abkürzungen müssen in einem
Abkürzungsverzeichnis erläutert werden, sofern es sich nicht um
allgemein bekannte Abkürzungen handelt.

\subsection{Kontrolle und
Formatierung}\label{kontrolle-und-formatierung}

Bewährt hat sich, wenn man den fertigen Entwurf mit etwas Abstand
betrachtet (z.B. mit einem zeitlichen Abstand von mindestens einem Tag).
Beim späteren Lesen sind folgende Punkte zu prüfen:

\subsubsection{Kontrolle des Inhalts, der Verständlichkeit und der
Sprache}\label{kontrolle-des-inhalts-der-verstuxe4ndlichkeit-und-der-sprache}

\begin{itemize}
\item
  Stimmen die gemachten Aussagen mit den eigentlichen Zielen überein?
\item
  Sind die eigentlichen Ergebnisse genügend gut hervorgehoben und
  stichhaltig begründet?
\item
  Ist die Gliederung des Hauptteils logisch?
\item
  Sind die Titel kurz und verständlich?
\item
  Sind die Übergänge von einem Thema zum anderen klar abgefasst?
\item
  Sind die Abschnitte zu lang oder zu kurz?
\item
  Unterstützen die gewählten Abbildungen und Tabellen den Text wirklich?
\item
  Sind Abbildungen und Tabellen passend beschriftet?
\item
  Ist die Satzstruktur klar und grammatikalisch richtig? (Kurze Sätze
  machen den Text leichter verständlich)
\item
  Sind alle Tippfehler korrigiert worden?
\item
  Stimmt die Fachterminologie? Ist die Wahl der Fachwörter treffend?
\item
  Sind die Quellenangaben vollständig?
\end{itemize}

Bevor die endgültige Formatierung vorgenommen wird, sollte der Bericht
einer anderen kompetenten Person unterbreitet werden, die konstruktive
Kritik anbringen kann. Sollten beim Leser irgendwo Unklarheiten
entstehen, ist dies in der Regel ein Indiz dafür, dass die Passagen
umzuschreiben sind.

\paragraph{Kontrolle der
Endformatierung}\label{kontrolle-der-endformatierung}

\begin{itemize}
\item
  Bei den Überschriften maximal drei (allerhöchstens vier) Stufen
  verwenden
\item
  Gut leserliche Schrift wählen
\item
  Schriftgrösse 10 -- 12
\item
  Nicht zu engen Zeilenabstand verwenden
\item
  Alle Nummerierungen und Querverweise überprüfen
\item
  Abstände überprüfen
\item
  Schriftgrössen und Schriftart überprüfen
\item
  Abbildungen an die geeignete Position bringen und allenfalls in der
  Grösse anpassen
\item
  Seitenumbrüche kontrollieren und gegebenenfalls neu definieren
\item
  Seitenangaben im Inhaltsverzeichnis überprüfen
\end{itemize}

Fremdsprachigen Studenten wird dringend empfohlen, den Technischen
Bericht von einer muttersprachigen Person korrigieren zu lassen.

Die Kontrolle des Textes in Bezug auf Inhalts-, Sprach- oder
Formatierungsfehler ist sehr wichtig. Oft wird dafür erfahrungsgemäß zu
wenig Zeit eingeplant. Nicht überarbeitete Texte erwecken einen
unseriösen Eindruck. Es ist schade, wenn aus zeitlichen Gründen eine
gründliche Überarbeitung des Berichts weggelassen wird. Der Bericht
präsentiert schließlich das eigentliche Resultat der Arbeit! Gute
Inhalte können bei einer schlechten Form eines Berichtes nicht
ausreichend zur Geltung kommen.

\subsection*{Referenzen}\label{referenzen}
\addcontentsline{toc}{subsection}{Referenzen}

\phantomsection\label{refs}
\begin{CSLReferences}{1}{0}
\bibitem[\citeproctext]{ref-boes2012}
Boes, R. 2012. \emph{Anforderungen und {H}inweise für das {V}erfassen
eines {T}echnischen {B}erichts und die {P}räsentation von {P}rojekt-,
{B}achelor-und {M}asterarbeiten}. ETH - Eidgenössische Technische
Hochschule Zürich, VAW - Versuchsanstalt für Wasserbau, Hydrologie und
Glaziologie.

\end{CSLReferences}




\end{document}
